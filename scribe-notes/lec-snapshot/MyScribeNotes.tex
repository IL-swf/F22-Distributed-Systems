%
% This is the LaTeX template file for lecture notes for EE 382C/EE 361C.
%
% To familiarize yourself with this template, the body contains
% some examples of its use.  Look them over.  Then you can
% run LaTeX on this file.  After you have LaTeXed this file then
% you can look over the result either by printing it out with
% dvips or using xdvi.
%
% This template is based on the template for Prof. Sinclair's CS 270.

\documentclass[twoside]{article}
\usepackage{graphics}
\setlength{\oddsidemargin}{0.25 in}
\setlength{\evensidemargin}{-0.25 in}
\setlength{\topmargin}{-0.6 in}
\setlength{\textwidth}{6.5 in}
\setlength{\textheight}{8.5 in}
\setlength{\headsep}{0.75 in}
\setlength{\parindent}{0 in}
\setlength{\parskip}{0.1 in}

%
% The following commands set up the lecnum (lecture number)
% counter and make various numbering schemes work relative
% to the lecture number.
%
\newcounter{lecnum}
\renewcommand{\thepage}{\thelecnum-\arabic{page}}
\renewcommand{\thesection}{\thelecnum.\arabic{section}}
\renewcommand{\theequation}{\thelecnum.\arabic{equation}}
\renewcommand{\thefigure}{\thelecnum.\arabic{figure}}
\renewcommand{\thetable}{\thelecnum.\arabic{table}}

%
% The following macro is used to generate the header.
%
\newcommand{\lecture}[4]{
   \pagestyle{myheadings}
   \thispagestyle{plain}
   \newpage
   \setcounter{lecnum}{#1}
   \setcounter{page}{1}
   \noindent
   \begin{center}
   \framebox{
      \vbox{\vspace{2mm}
    \hbox to 6.28in { {\bf EE 382N: Distributed Systems
                        \hfill Fall 2017} }
       \vspace{4mm}
       \hbox to 6.28in { {\Large \hfill Lecture #1: #2  \hfill} }
       \vspace{2mm}
       \hbox to 6.28in { {\it Lecturer: #3 \hfill Scribe: #4} }
      \vspace{2mm}}
   }
   \end{center}
   \markboth{Lecture #1: #2}{Lecture #1: #2}
   %{\bf Disclaimer}: {\it These notes have not been subjected to the
   %usual scrutiny reserved for formal publications.  They may be distributed
   %outside this class only with the permission of the Instructor.}
   \vspace*{4mm}
}

%
% Convention for citations is authors' initials followed by the year.
% For example, to cite a paper by Leighton and Maggs you would type
% \cite{LM89}, and to cite a paper by Strassen you would type \cite{S69}.
% (To avoid bibliography problems, for now we redefine the \cite command.)
% Also commands that create a suitable format for the reference list.
\renewcommand{\cite}[1]{[#1]}
\def\beginrefs{\begin{list}%
        {[\arabic{equation}]}{\usecounter{equation}
         \setlength{\leftmargin}{2.0truecm}\setlength{\labelsep}{0.4truecm}%
         \setlength{\labelwidth}{1.6truecm}}}
\def\endrefs{\end{list}}
\def\bibentry#1{\item[\hbox{[#1]}]}

%Use this command for a figure; it puts a figure in wherever you want it.
%usage: \fig{NUMBER}{SPACE-IN-INCHES}{CAPTION}
\newcommand{\fig}[3]{
			\vspace{#2}
			\begin{center}
			Figure \thelecnum.#1:~#3
			\end{center}
	}
% Use these for theorems, lemmas, proofs, etc.
\newtheorem{theorem}{Theorem}[lecnum]
\newtheorem{lemma}[theorem]{Lemma}
\newtheorem{proposition}[theorem]{Proposition}
\newtheorem{claim}[theorem]{Claim}
\newtheorem{corollary}[theorem]{Corollary}
\newtheorem{definition}[theorem]{Definition}
\newenvironment{proof}{{\bf Proof:}}{\hfill\rule{2mm}{2mm}}

% **** IF YOU WANT TO DEFINE ADDITIONAL MACROS FOR YOURSELF, PUT THEM HERE:

\begin{document}
%FILL IN THE RIGHT INFO.
%\lecture{**LECTURE-NUMBER**}{**DATE**}{**LECTURER**}{**SCRIBE**}
\lecture{3}{September 15}{Vijay Garg}{Patrick Sigourney}
%\footnotetext{These notes are partially based on those of Nigel Mansell.}

% **** YOUR NOTES GO HERE:

% Some general latex examples and examples making use of the
% macros follow.  
%**** IN GENERAL, BE BRIEF. LONG SCRIBE NOTES, NO MATTER HOW WELL WRITTEN,
%**** ARE NEVER READ BY ANYBODY.
\section{Global Snapshot}
Imagine a looking up at the sky and a flock of flying birds. You want to take a {\em global snapshot} of the sky in order to capture the state of all the birds.  You have a camera, but it is only able to photograph a small section of the sky in each image.  So you take a bunch of pictures of different parts of the sky in order to capture everything.  But as you're taking the pictures, the birds are flying around.  How do you capture an accurate snapshot?  There are two problems which must be overcome: 1) The double-counting of birds and 2) Missed counting of birds (birds in transit).


\section{Chandy-Lamport Snapshot Algorithm (1985)}
Developed by Leslie Lamport and K. Mani Chandry, the snapshot algorithm assumes interprocess communication occurs reliably, in FIFO (messages sent from A to B arrive in the same order they were sent), and the graph underlying the distributed processes is strongly connected.\\
The snapshot through the process timelines separates the timelines into "past" and "future", such that for all events which occurred in the "past" happened-before all events which occurred in the "future":  \\\\
(E, $\rightarrow$) is the set of events in "happened before"\\
A set G $\in$ E is in a consistent global state if $\forall(e,f): (f \in G) \wedge (e \rightarrow f) \Rightarrow e \in G$\\
(tl;dr: If f is in the 'past' and e happened before f, then e is also in the 'past'.  This resolves the double-counting problem).\\\\

To resolve the missed-counting problem, in addition to knowing the state of each process, we must also know the state of each channel.  We need a picture of the sender, the receiver, and what is in transit.



\subsection{Red and White}

The Chandy-Lamport algorithm is such:\\

1) All processes start as white.\\
2) Process A takes a local snapshot and turns red.\\
3) Process A sends a "marker" message on every outbound channel.\\
4) Process B receives the marker message, takes a local snapshot, turns red, it sends a marker on every outbound channel.\\
5) etc...\\

This is enough to capture the process states, but we also need to capture the state of each channel.\\

There are 4 possible message types, based on the color of the sender and receiver:\\
WW (white sending to white) - Nothing to do here.\\
RR (red sending to red) - Nothing to do here.\\
WR (white sending to red) - This is an "in transit" message and must be recorded.\\
RW (red sending to white) - Because the first thing a newly-red process does is send a marker to all other directly-connected processes, and because messages are assumed to be FIFO, there should NEVER be a message sent from a red process to a white process.  It can't happen.\\

For the WR messages:  These must be recorded by the receiving process to capture the state of the channel at the time of the snapshot.
  



\subsection{The Algorithm}
\begin{tabbing}
\underline{P$_{i}$} :\= \\
\> var Color: {white, red} initially white\\
\> var Closed[k]: array of bools, true once marker has been received from process k\\
\> var Channel[k]: array of linked lists of messages received from white process k\\
\\
\> def \= turnRed: if Color == white\\
\> \> record local state\\
\> \> Color := red\\
\> \> Send marker on all outgoing channels\\
\\
\> def recvMarker from k: \\
\> \> if color = white then: turnRed \\
\> \> closed[k] := true\\
\\
\> def recvMessage from k:\\
\> \> if color \= == red \&\& closed[k] == false:\\
\> \> \> Channel[k].append(message)\\
\\
P$_{i}$ is done when Color = red and ($\forall$ incoming channels c: closed[c] = True)\\
\\
Message complexity:\\
\> Number of overhead messages = number of channels in the graph\\
\> n nodes in completely connected graph = $(n^{2})$ messages\\
\\
\end{tabbing}
If FIFO is not assumed, the marker can also include the number of messages sent while white as a way for the receiver to determine whether all white messages have been received.

\end{document}





